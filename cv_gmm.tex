%%%%%%%%%%%%%%%%%%%%%%%%%%%%%%%%%%%%%%%%%
% Medium Length Graduate Curriculum Vitae
% LaTeX Template
% Version 1.1 (9/12/12)
%
% This template has been downloaded from:
% http://www.LaTeXTemplates.com
%
% Original author:
% Rensselaer Polytechnic Institute (http://www.rpi.edu/dept/arc/training/latex/resumes/)
%
% Important note:
% This template requires the res.cls file to be in the same directory as the
% .tex file. The res.cls file provides the resume style used for structuring the
% document.
%
%%%%%%%%%%%%%%%%%%%%%%%%%%%%%%%%%%%%%%%%%

%----------------------------------------------------------------------------------------
%	PACKAGES AND OTHER DOCUMENT CONFIGURATIONS
%----------------------------------------------------------------------------------------

\documentclass[margin, 11pt]{res} % Use the res.cls style, the font size can be changed to 11pt or 12pt here

\usepackage[usenames, dvipsnames]{xcolor}  % https://www.sharelatex.com/learn/Using_colours_in_LaTeX
\definecolor{mygray}{gray}{0.5}

\usepackage{helvet} % Default font is the helvetica postscript font
%\usepackage{newcent} % To change the default font to the new century schoolbook postscript font uncomment this line and comment the one above
\usepackage{hyperref}
\usepackage{url}
% costumize lists (itemize environments)
\usepackage{enumitem}

\usepackage[nodayofweek]{datetime}


% Remove page numbers since I can't center them properly
\usepackage{scrpage2}
%%\ifoot[]{}
\cfoot[]{}
%%\ofoot[\pagemark]{\pagemark}
\pagestyle{scrplain}

\setlength{\textwidth}{5.1in} % Text width of the document

\begin{document}

%----------------------------------------------------------------------------------------
%	NAME AND ADDRESS SECTION
%----------------------------------------------------------------------------------------

\moveleft.5\hoffset\centerline{\large\bf Guillermo Montero-Melis} % Your name at the top
 
\moveleft\hoffset\vbox{\hrule width\resumewidth height 1pt}\smallskip % Horizontal line after name; adjust line thickness by changing the '1pt'
 
\moveleft.5\hoffset\centerline{Max Planck Institute for Psycholinguistics, Wundtlaan 1, 6525 XD Nijmegen,
The Netherlands}
\moveleft.5\hoffset\centerline{\& Department of Linguistics, Stockholm University, 106 91 Stockholm, Sweden} % Your address
\moveleft.5\hoffset\centerline{%+46/(0)8 16 2919, 
\href{mailto:guillermo.monteromelis@mpi.nl}{guillermo.monteromelis@mpi.nl} /
\href{mailto:montero-melis@ling.su.se}{montero-melis@ling.su.se}}
\vspace*{.1in}
\moveleft.5\hoffset\centerline{\footnotesize {Updated: \today } }


%----------------------------------------------------------------------------------------

\begin{resume}

%%----------------------------------------------------------------------------------------
%%	OBJECTIVE SECTION%----------------------------------------------------------------------------------------
% 
%\section{OBJECTIVE}  
%
% Write something here if you have an objective in mind 

\section{\sc Research interests}
Language and thought, linguistic relativity,
conceptual representation, semantic typology,
second language acquisition, bilingualism,
cross-linguistic influence, 
mechanisms underlying language learning
\\



%----------------------------------------------------------------------------------------
%		EMPLOYMENT SECTION
%----------------------------------------------------------------------------------------

\section{\sc Employment}

{\bf Post-doctoral fellow} \hfill 2019--present\\ 
Max Planck Institute for Psycholinguistics, Nijmegen (Netherlands) and
Department of Linguistics at Stockholm University, Stockholm (Sweden)


{\bf Post-doctoral researcher} \hfill 2017--2018\\
Centre for Research on Bilingualism, Stockholm University, Sweden
 
{\bf PhD candidate} \hfill 2011--2017\\
Centre for Research on Bilingualism, Stockholm University, Sweden
 
 

%----------------------------------------------------------------------------------------
%	EDUCATION SECTION
%----------------------------------------------------------------------------------------

\section{\sc Education}

{\bf Stockholm University}, Stockholm, Sweden \hfill 2011--2017\\
Ph.D. in Bilingualism, June 2017
\vspace*{.05in}\\
Centre for Research on Bilingualism and Special Doctoral Programme in Language and Linguistics
\vspace*{.05in}\\
Dissertation title:  \href{http://su.diva-portal.org/smash/record.jsf?pid=diva2:1092276}{``Thoughts in Motion: The Role of Long-Term L1 and Short-Term L2 Experience when Talking and Thinking of Caused Motion''}
\vspace*{.05in}\\
Main advisors:  Emanuel Bylund, T.\,Florian Jaeger
\vspace*{.05in}\\
\textcolor{mygray}{Paternity leave: 01--04/2013, 12/2013--02/2014, 10/2014--05/2015 (c 15 months)}

%\vspace*{-.1in}
{\bf Stockholm University}, Stockholm, Sweden \hfill 2010--2011\\
Master of Arts in Education (Spanish and French)\\

\vspace*{-.1in}
{\bf Paris--Sorbonne University}, Paris, France \hfill 2002--2007\\
Master de recherche, Etudes Romanes –- Espagnol \\
Licence Espagnol, Mention Fran\c{c}ais Langue \'{E}trang\`{e}re \\
 

\section{\sc Further education \& training}

\begin{itemize}

\item Donders (f)MRI Tool-kit, Donders Centre for Cognitive Neuroimaging, Nijmegen, Netherlands, 3--7 June 2019

\item Professional Development Course on Teaching and Learning (7.5 ECTS), Stockholm Univ, Spring 2018

\item Functional magnetic resonance imaging: data analysis and experimental design (doctoral course, 1.5 ECTS), Department of Neuroscience, Karolinska Institute, Stockholm, Sweden, Nov 2017

\item Kavli Summer Institute in Cognitive Neuroscience, UC Davis / UC Santa Barbara, June--July 2017

\item Research stays in Florian Jaeger's lab, Dep. of Brain and Cognitive Sciences, Univ. of Rochester, USA; April--May 2014 and Oct 2015

\item Mathematics/statistics: 
Fundamental algebra and analysis (30 ECTS), Probability Theory (7.5 ECTS), Linear algebra (7.5 ECTS), Factor analysis with applications (7.5 ECTS)

\item Summer School in Bilingualism, Bangor University, July 2012
 
\item IMPRS NeuroCom Summer School, London, July 2011 
% International Max Planck Research School in Neuroscience and Communication

\item Foundation Research Methods in Psychology, Middlesex University London, 15 ECTS, June--July 2011

\end{itemize}



%\begin{itemize}[label=-]
%\item Probability Theory (7.5 ECTS). Dpt.~of Mathematics, Stockholm University, Spring 2015
%\item Fundamental algebra and analysis (30 ECTS). Dpt.~of Mathematics, Stockholm University, Fall 2013
%\item Linear algebra (7.5 ECTS). Faculty of Engineering and Sustainable Development, University of G\"{a}vle, Spring 2013
%\item Factor analysis with applications (7.5 ECTS). Transport and Location Analysis, Royal Institute of Technology, Spring 2012
%\item Elementary algebra and analysis (18 ECTS). Universit\'{e} Paris-Diderot Paris 7, 2003/2004
%\end{itemize}




%----------------------------------------------------------------------------------------
%	Scientific output
%----------------------------------------------------------------------------------------


\section{\sc Submitted / accepted /\\in press}

% If I rather want to create subsections, do the following, eg:
%\subsection{\small Publications}

%\vspace*{-.2in}

\begin{enumerate}

	\item \textbf{Montero-Melis, G.}, Paridon, J. van, Ostarek, M., \& Bylund, E. (2019). Does the motor system functionally contribute to keeping words in working memory? A pre-registered replication of Shebani and Pulvermüller (2013, Cortex). \emph{PsyArXiv}. \url{https://doi.org/10.31234/osf.io/pqf8k}

\end{enumerate}


\section{\sc Publications}

% If I rather want to create subsections, do the following, eg:
%\subsection{\small Publications}

%\vspace*{-.2in}

\begin{enumerate}

	\item \textbf{Montero-Melis, G.}, Isaksson, P., Paridon, J. van, \& Ostarek, M. (2020). Does using a foreign language reduce mental imagery? \emph{Cognition}, 196, 104134. \url{https://doi.org/10.1016/j.cognition.2019.104134}


	\item Ostarek, M., van Paridon, J., \& \textbf{Montero-Melis, G.} (2019). Sighted people’s language is not helpful for blind individuals’ acquisition of typical animal colors. \emph{Proceedings of the National Academy of Sciences}, 201912302. \url{https://doi.org/10.1073/pnas.1912302116}

	\item \textbf{Montero-Melis, G.}, \& Jaeger, T. F. (2019). Changing expectations mediate adaptation in L2 production. \emph{Bilingualism: Language and Cognition}, 1–16. \url{https://doi.org/10.1017/S1366728919000506}

%	\item \textbf{Montero-Melis, G.} (under revision). Speakers in motion: The role of speaker variability in motion encoding.

%	\item Melis, A., \textbf{Montero-Melis, G.}, Hay, S., Chater, N., \& Ludvig, E.\,A. (under revision). Justified and unjustified inequality: Fairness perceptions and dishonest behaviour.
	
	\item \textbf{Montero-Melis, G.} (2017). Thoughts in Motion: The Role of Long-Term L1 and Short-Term L2 Experience when Talking and Thinking of Caused Motion (Doctoral dissertation). Stocholm University, Centre for Research on Bilingualism, Stockholm. Retrieved from \url{http://su.diva-portal.org/smash/record.jsf?pid=diva2:1092276}

	\item \textbf{Montero-Melis, G.}, Eisenbeiss, S., Narasimhan, B., Ibarretxe-Antu\~{n}ano, I., Kita, S., Kopecka, A., L{\"u}pke, F., Nikitina, T., Tragel, I., Jaeger, T.\,F., \& Bohnemeyer, J. (2017). Talmy's framing typology underpredicts nonverbal motion categorization: Insights from a large language sample and simulations. \emph{Cognitive Semantics}, 3, 36--61. \url{http://doi.org/10.1163/23526416-00301002}

	\item \textbf{Montero-Melis, G.}, \& Bylund, E. (2017). Getting the ball rolling: the cross-linguistic conceptualization of caused motion. \emph{Language and Cognition}, 9(3), 446--472. \url{http://doi.org/10.1017/langcog.2016.22}

	\item \textbf{Montero-Melis, G.}, Jaeger, T.\,F., \& Bylund, E. (2016). Thinking is modulated by recent linguistic experience: Second language priming affects perceived event similarity. \emph{Language Learning}, 66(3), 636--665. \url{http://doi.org/10.1111/lang.12172}

	\item Athanasopoulos, P., Bylund, E., \textbf{Montero-Melis, G.}, Damjanovic, L., Schartner, A., Kibbe, A., Riches, N., \& Thierry, G. (2015). Two languages, two minds: Flexible cognitive processing driven by language of operation. \emph{Psychological Science}, 26(4), 518--526. \url{http://doi.org/10.1177/0956797614567509}

\end{enumerate}


\section{\sc Conference talks \& posters}
\begin{enumerate}

\item Isaksson, P., Ostarek, M., \& Montero-Melis, G. (2019). Does visual imagery impede reasoning? Evidence from L2 speakers. Poster presented at the 21st Meeting of the European Society for Cognitive Psychology (ESCoP 2019), Tenerife, Spain.

\item Montero-Melis, G. (2017, October) \textit{Thoughts in motion: Effects of language experience on motion cognition.} Oral presentation at the workshop Event Representations in Brain Language and Development, MPI for Psycholinguistics, Nijmegen, Netherlands.

\item Montero-Melis, G., \& Jaeger, T.\,F. (2017, September). \textit{Non-native (and native) adaptation to recent input during motion event lexicalization.} Paper presented at EuroSLA 27, University of Reading, Reading, UK.

\item Montero-Melis, G., \& Bylund, E. (2016, September). Getting the ball rolling: Cross-linguistic differences and commonalities in the representation of caused motion. Poster presented at AMLaP, Bilbao.

\item Montero-Melis, G., Buz, E., \& Jaeger, T.\,F. (2016, March). \textit{Does syntactic flexibility in production facilitate or inhibit planning?} Poster presented at the 29th Annual CUNY Conference on Human Sentence Processing, Gainesville, Florida.

\item Montero-Melis, G. (2016, March). \textit{What you('d) say is what you focus on: language effects on motion event conceptualization.} Poster presented at the pre-CUNY worshop on Events in Language and Cognition, Gainesville, Florida.

\item Montero-Melis, G. (2013, October). \textit{A data-driven approach to comparing semantic similarity of event descriptions across languages.} Paper presented at Investigating Semantics: Empirical and Philosophical Approaches, Ruhr-University Bochum, Germany.

\end{enumerate}



\section{\sc Invited talks}

\begin{enumerate}

\item Montero-Melis, G. (2018, June). \textit{Is the neural coding of concepts affected by the language we speak?}. Neurobiology of Language Department, Max Planck Institute for Psycholinguistics, Nijmegen, Netherlands.

\item Montero-Melis, G. (2017, November). \textit{Linguistic experience and conceptual representation of motion events}. Department of Linguistics, Uppsala University, Uppsala, Sweden.

\item Montero-Melis, G. (2017, October). \textit{A different manner of understanding motion events across languages? Effects of language experience on motion cognition}. Centre for Languages and Literature, Lund University, Lund, Sweden.

\item Montero-Melis, G. (2016, April). \textit{Ways of talking, ways of thinking: How language affects event representation}. Speech Production and Bilingualism Research Group (Cognition and Brain Research Unit), Universitat Pompeu Fabra, Barcelona, Spain.

\item Montero-Melis, G. (2015, October). \textit{Is learning a language a path towards new manners of thinking?}. Department of Brain and Cognitive Sciences (HLP Lab), University of Rochester, NY.

\item Montero-Melis, G. (2015, October). \textit{Is learning a language a path towards new manners of thinking?}. Department of Linguistics (Semantic Typology Lab), University at Buffalo, NY.

\item Montero-Melis, G. (2015, May). \textit{Getting the ball rolling: Cross-linguistic differences in the representation of caused motion}. Department of Experimental Psychology (ConCat Research group), University of Leuven, Belgium.

\item Montero-Melis, G. (2014, April). \textit{Cross-linguistic differences in event descriptions and similarity judgments: Caused motion events in Spanish and Swedish}. Department of Linguistics, University of Rochester, NY.

\item Montero-Melis, G. (2014, April). \textit{Cross-linguistic differences in event descriptions and similarity judgments: Caused motion events in Spanish and Swedish}. Department of Linguistics (Semantic Typology Lab), University at Buffalo, NY.

\item Montero-Melis, G. (2014, April). \textit{Cross-linguistic differences in similarity judgments: Caused motion events in Spanish and Swedish}. Psychology Department (Experience and Cognition Lab), University of Chicago, IL.

\end{enumerate}


% Another way of displaying it, with talks, etc. as main sections, instead of subsections:

%\section{\sc Refereed Conference \\Talks}
%%\begin{enumerate}
%%	\item 
%Montero-Melis, G. 2013. A data-driven approach to comparing semantic similarity of event descriptions across languages. \textit{Investigating Semantics: Empirical and Philosophical Approaches}, Ruhr-University Bochum, October 2013 
%%\end{enumerate}
%
%
%\section{\sc Colloquium Talks}
%
%\begin{enumerate}
%%	\item 
%
%\item Montero-Melis, G. 2014 (April). TBA. Brain and Cognitive Sciences, University of Rochester, NY.
%
%\item Montero-Melis, G. 2014 (April). \emph{Assessing within-language variation in motion event descriptions -- a computational approach.} Department of Linguistics, University at Buffalo, NY.
%
%\item Montero-Melis, G. 2014 (April). TBA. Psychology Department, University of Chicago.
%
%\end{enumerate}


%----------------------------------------------------------------------------------------
%	FUNDING/GRANTS
%----------------------------------------------------------------------------------------


\section{\sc Funding}

%Travel grant from Helge Ax:son Johnsons Stiftelse for participation at ESCoP conference (Tenerife, Spain), 15\,000~SEK (1400~EUR), September 2019.

Travel grant from Wenner-Gren Foundation for participation at ESCoP conference (Tenerife, Spain), 8\,000~SEK (750~EUR), September 2019.

Three-year International Postdoctoral grant from the Swedish Research Council for the project ``Do concepts in our brain depend on the language we speak?'', 3.15 million~SEK (315\,000~EUR), 2019--2021.

Travel grant from the IDO foundation for a research stay at the MPI for Psycholinguistics (Nijmegen, NL) during Spring 2018, 13\,840~SEK (1\,400~EUR), November 2017.

Travel grant from Kungliga Vitterhetsakademien for participation at the workshop Event Representations in Brain Language and Development at the MPI for Psycholinguistics (Nijmegen, NL), 10\,000~SEK (1\,040~EUR) October 2017.

Travel grant from Stiftelsen Elisabeth och Herman Rhodins minne for participation at EuroSLA 27 (Univ. of Reading), 11\,000~SEK (1\,130~EUR), 2017

Travel grant from Gunvor och Josef An\'{e}rs stiftelse for participation at the Kavli Summer Institute in Cognitive Neuroscience 2017 (UC Davis, UC Santa Barbara), 13\,000~SEK (1\,330~EUR), 2017

Travel grant from Kungliga Vitterhetsakademien for participation at CUNY 2016 (Univ.\,of Florida), 17\,000~SEK (1\,820~EUR), 2016

Travel grant from Helge Ax:son Johnsons Stiftelse for research stay at Univ.\,of Rochester (USA), 17\,000~SEK (1\,820~EUR), 2015

Travel grant from G\aa l\"{o}stiftelsen for study trip to the USA (Univ.\,of Rochester, Univ.\,at Buffalo, Univ.\,of Chicago), 18\,000~SEK (2\,000~EUR), 2014
 


%----------------------------------------------------------------------------------------
%	COLLABORATIONS
%----------------------------------------------------------------------------------------


%\section{\sc Collaborations}





%----------------------------------------------------------------------------------------
%	OTHER MERITS
%----------------------------------------------------------------------------------------


\section{\sc Reviewing}

Ad hoc reviewer: 
\textit{Bilingualism: Language and Cognition},
\textit{Cognitive Linguistics},
\textit{eLife},
\textit{International Journal of Multilingualism},
John Benjamins series \textit{Language Learning and Language Teaching},
\textit{Language and Cognition},
\textit{Language Learning},
\textit{Lingua},
\textit{Linguistic Approaches to Bilingualism},
National Science Foundation (NSF)



%----------------------------------------------------------------------------------------
%	UNIVERSITY TEACHING
%----------------------------------------------------------------------------------------


\section{\sc Teaching}

\begin{itemize}

\item \textit{Statistics and statistical methods} (individual reading course). Doctoral course at Centre for Research on Bilingualism, Stockholm Univ., Spring 2018

\item \textit{Theory and Method in Second Language Research} (shared with other instructor). Undergraduate course at Centre for Research on Bilingualism, Stockholm Univ., Spring 2018

\item \textit{Data analysis for master's students}. Workshop in three sessions, Stockholm Univ., Spring 2016

\item \textit{Second language acquisition}. Undergraduate course at Centre for Research on Bilingualism, Stockholm Univ., Fall 2015

\item \textit{Statistics and statistical methods} (shared with other instructor). Doctoral course at Centre for Research on Bilingualism, Stockholm Univ., Spring 2015

\item \textit{Second language research}. Undergraduate course at Centre for Research on Bilingualism, Stockholm Univ., Fall 2013 

\end{itemize}


\section{\sc Mentorship}

\begin{itemize}

\item Teun van Gils, PhD advisor (co-promotor), IMPRS Max Planck Institute for Psycholinguistics; Fall 2019--ongoing

\item Pia Järnefelt, MA thesis advisor, Stockholm University; Spring 2019

\item Petrus Isaksson, undergraduate thesis advisor, Stockholm University; fall 2018

\item Gaston Prieto, undergraduate thesis advisor, Stockholm University; fall 2018

\end{itemize}

%----------------------------------------------------------------------------------------
%	OTHER ACTIVITIES
%----------------------------------------------------------------------------------------


\section{\sc Departmental Service \& other academic activities}

\begin{itemize}

\item Statistical support for researchers at the departmental level. 
Fall 2013--2018

\item Responsible for statistical analyses in Tisus (Test in Swedish for university studies). 
Fall 2013--2016

%\section{\sc Other \\Academic Activities}

\item Launcher and organizer of the weekly psycholinguistic lab meetings at Centre for Research on Bilingualism (started January 2016)

\item Launcher and organizer of the Data Analysis and Statistics Study Group (2014--2015)

\item Organized graduate student workshop \emph{Challenges in Quantitative Semantics}, together with Jussi Karlgren and Susanne Vejdemo (Stockholm University, June 2013)

\end{itemize}


%----------------------------------------------------------------------------------------
%	OUTREACH / MEDIA COVERAGE
%----------------------------------------------------------------------------------------

\section{\sc Outreach \\and Media}

\begin{itemize}

\item Popular science talk on language and thought at \href{https://cultura.cervantes.es/estocolmo/es/-hablas-como-piensas--la-relacion-entre-el-lenguaje-y-el-pensamiento/122695}{Cervantes Institute, Stockholm}; 2 October 2018.
Video available at \url{https://www.youtube.com/watch?v=5uJ3vpFz1qU}

\item Popular science talk on whether learning a new language reshapes our way of thinking, for teachers of Spanish as a foreign language, organized by the Centre for Professional Development and Internationalisation in Schools (Uppsala University, Sweden), 3 February 2018.

\item Josephson, O. (2017, June 21). Uppfattar svensk- och spansktalande v{\"a}rlden olika? [Do Swedish and Spanish speakers perceive the world differently?] \emph{Svenska Dagbladet}. Retrieved from \url{https://www.svd.se} (A column on my PhD dissertation in one of the major Swedish daily newspapers)

\item Skolverket, \& Sayehli, S. (2017, June 21). Spr{\aa}k och tanke -- en avhandling om spr{\aa}klig relativism och flerspr{\aa}kighet [Language and thought: a dissertation on linguistic relativity and multilingualism]. Retrieved 22 June 2017, from \url{https://www.skolverket.se} (A summary of my PhD dissertation by The Swedish National Agency for Education)

\item Athanasopoulos, P. (2015, April 27). \href{https://www.theguardian.com/commentisfree/2015/apr/27/world-view-learn-another-language}{Think your world view is fixed? Learn another language and you'll think differently.} \emph{The Guardian}. (Opinion piece based on Athanasopoulos, Bylund, Montero-Melis et al., 2015.)

\end{itemize}


%----------------------------------------------------------------------------------------
%	OTHER PROFESSIONAL EXPERIENCE SECTION
%----------------------------------------------------------------------------------------
 
\section{\sc Experience Outside Academia}

\textsl{Language teacher:} \\
Kungstensgymnasiet (secondary school), Stockholm: Spanish  and French. 2009--2010 \\
Folkuniversitetet (adult education), Stockholm: Spanish and French. 2009 \\
Centro Cultural Rafael de Le\'{o}n, Madrid (adult education): German. 2008 \\ 
Ministerio de Medio Ambiente (adult education), Madrid: German. 2007--2008 \\ 


%----------------------------------------------------------------------------------------
%	ADMINISTRATION AND ORGANIZATION
%----------------------------------------------------------------------------------------

%\section{\sc Administration and Organization}



%----------------------------------------------------------------------------------------
%	Languages
%----------------------------------------------------------------------------------------

\section{\sc Languages}

Spanish: First language \\
German: C1 (primary and secondary school in Berlin and Luxembourg)\\
French: C1 (secondary school in Luxembourg, university studies in Paris)\\
English: C1\\
Portuguese (Brazil): B1\\
Swedish: B2\\


\section{\sc Programming and software}
R, SPSS, E-Prime, Python, SQL




%%----------------------------------------------------------------------------------------
%%	COMPUTER SKILLS SECTION
%%----------------------------------------------------------------------------------------
%
%\section{COMPUTER \\ SKILLS} 
%
%{\sl Languages \& Software:} 
%COBOL, IFPS, Focus, Megacalc, Pascal, Modula2, C, APL, SNOBOL, FORTRAN, LISP, SPIRES, BASIC, VSPC Autotab, IBM 370 Assembler, Lotus 1-2-3. \\
%{\sl Operating Systems:} MTS, TSO, Unix. 
 


%----------------------------------------------------------------------------------------
%	EXTRA-CURRICULAR ACTIVITIES SECTION
%----------------------------------------------------------------------------------------

%\section{\sc Miscellaneous} 
%
%Accompanying texts for photography exhibition \emph{Paradoxe} by photographers Arantxa Hurtado and Irene Hurtado, Luxembourg, 2008 \\
%Practiced the Brazilian martial art Capoeira since the age of sixteen \\
%Trumpet player in different ensembles, 2002--2008 \\
%Bike ride Madrid--Stockholm with my wife, June--September 2008\\
%Treasurer of the non-profit organization Kollektivhus Fullersta Backe (housing cooperative), since 2013

%----------------------------------------------------------------------------------------

\end{resume}
\end{document}